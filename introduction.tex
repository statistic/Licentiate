\section{Introduction}
Recent advances in information technology makes it possible to create an awareness of the environment, which includes to detect and respond to the presence of a human. This might be done due to variety of reasons, such as security applications like intruder detection and collision avoidance, safety applications such as avalanche and rubble search and rescue missions, police raid operations, driving assistance, elderly care and living assistance for Alzheimer patients.
Depending on the application there are different requirements of the detection and sensing. Examples include pure determination of presence or specifying the location, tracking or identification. This theses focuses on the determination of presence and localization of human in enclosed environments. 

While machineries are becoming important in the industry, new requirements emerge, one is to allow humans to safely collaborate with robots and machines. This will increase the efficiency of sites and reduce accidents and injuries. Techniques such as infrared detectors\cite{InfraredHumanDetection}, vision based systems\cite{VisualSurveillanceBoult},\cite{Visualhumandetection}, vibration and seismic waves\cite{VibrationTracking}, acoustics\cite{AcousticsHumanDetection} and radar sensors\cite{YarovoyUWBHumanDetection} are used as a solution to human detection problem.

Some systems are built based on the fusion of these sensors to take advantage of each sensor strength to increased confidence of detection and decrease false alarms.
In\cite{RadarVisionfusion} Milch and Behrens used radar-vision fusion for detecting pedestrians on-board a moving vehicle. Radar sensor is used to generate a target list or hypotheses for presence of pedestrians. In the next step vision system is used to prove the hypotheses, if the target is a pedestrian. 

Human detection using radar has advantages such as the ability to function in all lightning conditions whereas visual, infrared and laser sensors are prone to fog, smoke, dirt, environment’s lighting conditions or temperature changes. In addition visual, infrared, and seismic sensors need to be placed in close proximity to the target whereas radar senors depend on frequency of the operation can function up to several hundred meters.

The human detection with radar has two main tasks: first the target shall be detected then it shall be decided if the detected target is a human or not. It has been different approaches in how to discriminate a human target from the other detected targets. In many of sensor fusion solutions that radar is a part of, other types of sensors are responsible for the determination whether the target is a human or not because the radar is considered to be incapable of making such decisions. The problem with this approach is that the benefits of radar sensors are not fully taken advantage of. This thesis is aiming at using the radar sensor as a stand alone solution for the whole human detection problem. This means that both detection and discrimination of the human from other targets is performed by the radar sensor. 
  
\section{Background}

Radar (RAdio Detection And Ranging) uses electromagnetic waves from a transmitter and when it interacts with an object a part of it reflects, absorbs and scatters depend on the frequency of wave, the shape and material of the object and the orientation of the incident wave. The reflected wave is different from the transmitted wave in frequency, time delay and amplitude. The frequency shift is the result of the Doppler effect due to the relative speed of the radar and the target. The time delay is due to the time it takes for the waves to travel from the radar transmitter to the object and back again. Attenuation of the signal is a result of path loss over the traveling distance by the wave. The received signal power by the radar is defined by the radar equation \ref{eq:radareq}.
\begin{equation}\label{eq:radareq}
	P_{r}= \frac{P_{t} G_{t}}{4\pi R^{2}}\times\frac{\sigma}{4\pi R^{2}}\times A_{e}
\end{equation}

The equation is written based on the product of three factors: The first fraction in equation \ref{eq:radareq} represents the power density at distance R from a radar that transmits with the power of $P_{t}$ from an antenna with the gain of $G_{t}$. The second fraction contains $\sigma$ which is radar cross section of the target and is the measure of the energy reflected from the target back to the radar. The last term in equation \ref{eq:radareq} is  $A_{e}$ which is the receiving antenna effective area\cite{skolnik2008radar}. 

%The difference between traditional radio technolgy and ultra wide band
The majority of traditional radar system are based on a narrow band signal modulated on a sinusoidal carrier wave. The amount of information that this radio systems can carry is limited due to limited bandwidth. To increase the information amount wide band and Ultra Wide Band(UWB) signals shall be chosen. This is possible due to usage of the narrower pulse in order of nano seconds, which will increase the accuracy of target range and the ability to detect smaller targets and more detailed information about them. In addition it will reduce the passive interference from for example rain and other particles due to relative reduction of scattering cross section of interference to the target\cite{taylor2000ultra}.

%Narrow band radar systems use sinusoidal signals which tend to keep their shape during signal conversions such as addition and differentiation. Because ultra wideband signals has different waveform the shape can change during signal conversion.(This part shall be added later in the licentiate)

%why UWB is better that other radar systems
UWB technology is a promising technology due to high spatial resolution and obstacle penetration capabilities. Over the past two decades the advancements in electronics made it possible to develop UWB radar system with advantages to other conventional technologies\cite{UWBHussain}.

Based on Federal Communications Commission (FCC) definition a signal can be defined as UWB signal if the fractional band with is greater than 0.2 where fractional bandwidth is defined in equation \ref{FracBW}.
\begin{equation}\label{FracBW}
	B_{f} = 2\frac{f_{H}-f_{L}}{f_{H}+f_{L}}
\end{equation}
%Some active localization system require the person carry a radio-frequency tag
%The disadvantage is that the data is not possible to be visualized for a human eye contra vision system
%detection relies on the back-propagation of the signal from the body
Different types of UWB radar are developed and based on the excitation wave-form they are divided to different categories. Three of the most prominent categories are Frequency Modulated Continuous Wave (FMCW), pulse and M-sequence. In this research we have access to a commercially available M-sequence UWB radar which is developed in Radarbolaget. Radarbolaget is positioned in G\"{a}vle, Sweden developing radar
systems primarily for real time, through wall monitoring of heating furnaces \footnote{http://www.radarbolaget.com}.To be able to understand the system constraint a comparison study between UWB M-sequence and pulse is done in paper A. The pulse radar belonged to Time Domain\textsuperscript{TM}, one of the leading companies in developing UWB radar and communication systems \footnote{http://www.timedomain.com}. Sachs et. al. compared pulse and M-sequence for UWB technology in\cite{sachs2003stimulation}. Based on technical specifications of each system it is reasoned that the pulse based
UWB systems are desirable in applications with simple data interpretation and low power consumption versus the M-sequence based approach needs more complicated data processing but provides highly stable data.


%As humans are made of 70\% water a good part of this waves reflect back to the receiver. 
%The frequency of th radar also can determine the size of the detectable object. The lower the wave length the smaller the detectable object. traditional radar devices send a wave with a small frequency band and in the receiver the reflected signal can be processed and filtered by that specific frequency.
%the goal of research questions or 
%research challenges are sensing noise, environment variation\cite{Teixeira2010}.
%different types of clothing, hats, backpacks, purses, and\cite{Dogaru2007} \cite{Yamada2005}
.