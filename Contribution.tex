\chapter{Contribution}
\section{Summary of The Research Work throughout Paper A-C}
\section{Contribution of Included Papers}
\label{papers}
\begin{itemize}
	\item \textbf{Paper A} \textit{Experimental Comparison Study of UWB Technologies for Static Human Detection}\\*
	Melika Hozhabri, Magnus Otterskog, Nikola Petrovic and Martin Ekstr\"{o}m\\*
	IEEE International Conference on Ubiquitous Wireless Broadband (ICUWB 2016), Nanjing, China.
	
	Abstract and Contribution:This paper compares two dominant Ultra Wide Band(UWB) radar technologies Impulse and M-sequence for static human being detection in free space. The hardware and software platform for each system is described separately. These two radar platform performances are tested in real conditions and the results show that M-sequence UWB radar is better suited for detecting the static human target in larger distances.
	
	Author's contribution:
	I have been the main author of this paper and a major part of the idea was mine. I have planned and performed the measurements with help of Radarbolaget that provided the hardware. I wrote most of the paper and analyzed the results.
	
	\item \textbf{Paper B} \textit{Study of Environment Effect on Detection of
		Walking Human by M-Sequence UWB Radar }\\*
		Melika Hozhabri, Per-Olov Risman and Nikola Petrovic
		2016 IEEE Conference on Antenna Measurements \& Applications (CAMA)
		
	Abstract and Contribution: This paper presents an experimental comparison study of human movement and presence detection in different environments using ultra-wide-band (UWB) M-Sequence radar. The benchmarking measurements are made in an anechoic chamber and repeated in an open office environment. The wave forms of the background noise and scattered amplitudes of a human body are measured and compared. A set of detection algorithms and filters which are developed to track the human movement and presence is presented and the tracking results in these two environments are compared to each other.

	Author's contribution:
	I have been the main author of this paper. I have planned and performed the measurements with help of Delta that provided the semi-anechoic chamber. I wrote most of the paper and partly analyzed the results. I partly developed the signal processing algorithms and filters.  
	\item\textbf{Paper C} \textit{Comparison of UWB Radar Backscattering by the
    Human Torso and a Phantom}\\*
		Melika Hozhabri, Per-Olov Risman and Nikola Petrovic
		2018 IEEE Conference on Antenna Measurements \& Applications (CAMA)
		
	Abstract and Contribution: An Ultra Wide Band (UWB) radar is used to measure the backscattering of a human and a human phantom. The choice of material and shape for the human phantom is discussed. The dielectric properties of the material (wet sand) used in the experiment are measured by a retromodeling technique and also calculated by mixture formulas. The appropriate frequency choice for the application is discussed.
	
	Author's contribution:
	I have been the main author of this paper. I have planned and performed the measurements with help of Radarbolaget in G{\"a}vle University that provided the radar hardware and . I wrote most of the paper and partly analyzed the results.\todo{change this abit, write about what PO did?}
	
\end{itemize}
To be able to understand the system constraint a comparison study between UWB M-sequence and pulse is done in paper A. The pulse radar belonged to Time Domain\textsuperscript{TM}, one of the leading companies in developing UWB radar and communication systems \footnote{http://www.timedomain.com}. Sachs et. al. compared pulse and M-sequence for UWB technology in \cite{sachs2003stimulation}. Based on technical specifications of each system it is reasoned that the pulse based UWB systems are desirable in applications with simple data interpretation and low power consumption versus the M-sequence based approach needs more complicated data processing but provides highly stable data.